\documentclass[10pt,letterpaper,fleqn]{article}

\usepackage[utf8]{inputenc}
\usepackage[spanish,es-nodecimaldot]{babel}
\usepackage{amsmath}
\usepackage{amssymb}
\usepackage{multicol}
\usepackage{graphicx}
\usepackage{amsthm}

\usepackage[dvipsnames]{xcolor}
\usepackage[most]{tcolorbox}

\usepackage{tabu}

\usepackage{pgfplots}
\pgfplotsset{width=10cm,compat=1.9}

\usepackage{mathtools}
\usepackage{tikz}
\usetikzlibrary{trees,positioning,babel}

\usepackage[top=1in, bottom=1in, left=1in, right=1in]{geometry}


\begin{document}

% Portada
\begin{titlepage}
    \centering

    {\scshape\LARGE Universidad Nacional Autónoma de México \par}

    \vspace{1cm}
    
    {\scshape\Large Facultad de Ciencias\par}

    \vspace{1.5cm}

	% Logo de la UNAM.
    \begin{center}
        \includegraphics[scale=.17]{../../assets/img/logo.png}
    \end{center}

    \vspace{1 cm}

    {\LARGE Tarea 2 \par}
    
    \vspace{.3cm}
    
    {\huge\bfseries Matemáticas Aplicadas para las Ciencias III \par}

	\vspace{1.3cm}

    \large{\itshape{Johann Ramón Gordillo Guzmán}} \small{ - 418046090} \\
    
    \vspace{1cm}
	
	\large{\itshape{Luis Erick Montes Garcia}} \small{ - 419004547} \\    
    
	\vspace{1cm}    
    
    \large{\itshape{Ledesma Rincon Orlando}} \small{ - 419003234} \\

	\vspace{1cm}    
    
    \large{\itshape{Raymundo Méndez García}} \small{ - 113001958} \\

	\vspace{1cm}    
    
    \large{\itshape{Alex Gerardo Fernández Aguilar}} \small{ - 314338097} \\
    
    \vspace{0.3cm}

    \vfill

    Tarea presentada como parte del curso de
    \textbf{Matemáticas para las Ciencias Aplicadas III}
    impartido por el profesor \textbf{Zeús Alberto Valtierra Quintal}. \par
    \vspace{0.2cm}
    {\large 20 de Septiembre del 2019\par}
    \vspace{0.3cm}
    \footnotesize{\textbf{Link al código fuente:} https://github.com/JohannGordillo/Mates-lll-Tarea-2}
\end{titlepage}


    \begin{enumerate}

        %Ejercicio 1%
        \item Sea $S^* = (0,1]$ x $[0, 2\pi)$ y sea $T(r, \theta) = (r cos \theta, r sin \theta)$.\\
        Hallar la imagen del conjunto $S$.\\
        Demostrar que $T$ es inyectiva en $S^*$.\\\\
        Primero hallemos la imagen:\\
        $\lbrace (x, y) \in R^2 | \exists (r, \theta) \in S^* (T(r, \theta) = (x,y)) \rbrace$\\\\
        Sea $(x, y) = (r cos \theta, r sin \theta)$\\\\
        Entonces:\\
        $x = r cos \theta$\\
        $y = r sin \theta$\\\\
        Luego, 
        \begin{equation*}
        \begin{split}
        x^2 + y^2 & = r^2 cos^2 \theta + r^2 sin^2 \theta \\
        & = r^2 (cos^2 \theta + sin^2 \theta) \\
        & = r^2 (1) \\
        & = r^2  
        \end{split}
        \end{equation*}
        Sabemos que $0 < r \leq 1$, así que $0 < r^2 \leq 1$.\\\\
        $\therefore 0 < x^2 + y^2 \leq 1$\\\\
        Es decir, $x^2 + y^2 \leq 1$ y $x^2 + y^2 > 0$.\\\\
        Sabemos que $x^2 + y^2 = 0 \Leftrightarrow x = 0 = y$, por lo que excluimos de la imagen al origen $(0, 0)$.\\\\
        Finalmente, $x^2 + y^2 = 1$ es el disco unitario, por lo que el conjunto S serán los puntos del disco unitario exceptuando al origen. Falta únicamente probar que T es una inyección.\\
        \begin{proof}
        \vspace{0.2cm}
        Sean $(r,\theta), (r', \theta') \in S^*$.\\\\
        Supongamos que $T(r,\theta) = T(r', \theta')$.\\\\
        $\Rightarrow (r cos \theta, r sin \theta) = (r' cos \theta', r' sin \theta')$.\\
        $\therefore r cos \theta = r' cos \theta'$ y además $r sin \theta = r' sin \theta'$.\\\\
        Esto ocurre $\Leftrightarrow (r = r')$ o bien si $r = r'$ y $\theta = \theta' + 2 \pi k$ con $k \in Z$\\\\
        Al ser $0 < r \leq 1$, $r = r'$.\\\\
        Además, $0 \leq \theta < 2 \pi$, por lo que $k = 0$ y $\theta = \theta'$.\\\\
        $\therefore (r,\theta), (r', \theta')$\\\\
        $\therefore T$ es una inyección.\\
		\end{proof} 
	
		\newpage
        %Ejercicio 2%
        \item Sea $D^* = [0,1]$ x $[0, 1]$ y defínase $T$ en $D^*$ mediante $T(u, v) = (-u^2 - 4u, v).$\\
        Hallar la imagen $D$.\\
        ¿Es $T$ inyectiva?\\
        
        Tenemos que $D^* = [0,1]$ x $[0,1]$.\\
        Podemos analizar la frontera de la región y ver cómo se comporta la transformación al ser aplicada sobre éstas curvas:\\\\
        $c_1(t) = (t, 0)$\\
        $c_2(t) = (1, t)$\\
        $c_3(t) = (t, 1)$\\
        $c_4(t) = (0, t)$\\\\
        Al aplicar la transformación obténemos las rectas:\\
        $\gamma_1(t) = T(c_1(t)) = T(t, 0) = (-t^2 + 4t, 0)$\\
        $\gamma_2(t) = T(c_2(t)) = T(1, t) = (-1 + 4, t) = (3, t)$\\
		$\gamma_3(t) = T(c_3(t)) = T(t, 1) = (-t^2 + 4t, 1)$\\        
        $\gamma_4(t) = T(c_4(t)) = T(0, t) = (0, t)$\\
        con $t \in [0,1]$.\\
        $\Rightarrow \gamma_1$ es la recta del punto $(0,1)$ al punto $(3,1)$.\\
        $\Rightarrow \gamma_2$ es la recta del punto $(0,0)$ al punto $(0,1)$.\\
        $\Rightarrow \gamma_3$ es la recta del punto $(3,0)$ al punto $(3,1)$.\\
        $\Rightarrow \gamma_4$ es la recta del punto $(0,0)$ al punto $(3,0)$.\\
        Lo que obténemos son cuatro rectas que forman la región $D = [0, 3]$ x $[0, 1]$\\\\       
        Únicamente falta ver que la transformación sea una inyección. Analizando el comportamiento de la transformación y las regiones $D^*$ y $D$, podemos intuir que sí lo es. Pero hay que dar una demostración formal:
        \begin{proof}
        \vspace{0.2cm}
        Sean $(u,v), (u', v') \in D^*$.\\\\
        Supongamos que $T(u, v) = T(u', v')$.\\\\
        $\Rightarrow (-u^2 + 4u, v) = (-(u')^2 + 4u', v')$.\\
        $\therefore -u^2 + 4u = -(u')^2 + 4u'$ y además $v = v'$.\\\\
        Como $v = v'$, falta probar que $u = u'$.\\
        Despejando: $-u^2 + 4u + (u')^2 - 4u' = 0$\\
        Aplicando la fórmula general para ecuaciones de segundo grado con variables\\
        $a = 1$\\
        $b = -4$\\
        $c = -u^2 + 4u$\\
        obténemos:\\
        $u' = \frac{4 \pm \sqrt{16 - 4(-u^2 + 4u)}}{2} = 2 \pm \frac{\sqrt{16 + 4 u^2 - 4u}}{2} = 2 \pm \frac{sqrt{4(u-2)^2}}{2}$\\\\
        $\therefore u' = u$ o $u' = 4 - u$.\\\\
        Pero sabemos que $u', u \in [0,1]$, por lo que no es posible que $u' = 4-u$\\
        $\therefore u = u'$\\
        $\therefore (u, v) = (u', v')$\\\\
        $\therefore T$ es una inyección.\\
		\end{proof} 
        
        %Ejercicio 3%
        \item Sea $D^* = [0,1]$ x $[0, 1]$ y defínase $T$ en $D^*$ mediante $T(x^*, y^*) = (x^*y^*, x^*).$\\
        Hallar la imagen $D$.\\
        ¿Es $T$ inyectiva?\\
        Si no lo es, ¿se puede quitar un subconjunto a $D^*$ para que $T$ sea inyectiva?\\\\
        Tenemos que $D^* = [0,1]$ x $[0, 1]$, que es la misma región que en el ejercicio anterior, por lo que obténemos las mismas rectas en la frontera:\\\\
        $c_1(t) = (t, 0)$\\
        $c_2(t) = (1, t)$\\
        $c_3(t) = (t, 1)$\\
        $c_4(t) = (0, t)$\\\\
        Al aplicar la transformación obténemos las rectas:\\
        $\gamma_1(t) = T(c_1(t)) = T(t, 0) = (t(0), t) = (0, t)$\\
        $\gamma_2(t) = T(c_2(t)) = T(1, t) = (1(t), t) = (t, 1)$\\
		$\gamma_3(t) = T(c_3(t)) = T(t, 1) = (t(1), t) = (t, t)$\\        
        $\gamma_4(t) = T(c_4(t)) = T(0, t) = (0(t), 0) = (0, 0)$\\
        con $t \in [0,1]$.\\\\
        $\Rightarrow \gamma_1$ es el $eje-y$.\\
        $\Rightarrow \gamma_2$ es la recta $y = 1$.\\
        $\Rightarrow \gamma_3$ es la recta $y = x$.\\
        $\Rightarrow \gamma_4$ es el origen $(0, 0)$.\\
        Lo que obténemos son tres rectas que forman la región triángular $D$ con vértices $(0,0)$, $(0,1)$ y $(1, 1)$.\\\\   
        Únicamente falta ver que la transformación sea una inyección. Analizando el comportamiento de la transformación y las regiones $D^*$ y $D$, podemos intuir que no lo es, ya que todos los puntos de la recta $c_4$ son mapeados al origen por la transformación $T$.\\\\
        Sin embargo, podemos hacer que la transformación sea inyectiva eliminando únicamente a la recta $c_4$ del dominio de la transformación.\\\\           

        %Ejercicio 4%
        \item Demostraremos que $$ T(D^*)=\{ x\in\mathbb{R}^3 \mid |x|\leq 1 \}$$. Para esto vemos que 
        	\begin{equation*}
        	\begin{split}
        		T(p,\phi,\theta) &= (p\sin\phi\cos\theta, p\sin\phi\sin\theta, p\cos\phi) \\
        						 &= p(\sin\phi\cos\theta, \sin\phi\sin\theta, \cos\phi) \\
				||T(p,\phi,\theta)|| &= p ||(\sin\phi\cos\theta, \sin\phi\sin\theta, \cos\phi) || \\
								 &= p\sqrt{\sin^2\phi\cos^2\theta + \sin^2\phi\sin^2\theta + \cos^2\phi} \\ 
								 &= p\sqrt{\sin^2\phi(\cos^2\theta + \sin^2\theta) + \cos^2\phi} \\ 
								 &= p\sqrt{\sin^2\phi + \cos^2\phi} \\
								 &= p\sqrt{1}
								 = p \in \left[0,1\right] \\
								 & \therefore ||T(p,\phi,\theta)|| \in \left[0,1\right] \\
								 & \therefore T(p,\phi,\theta) \in \{ x\in \mathbb{R}^3 | |x| \leq 1 \}
        	\end{split}
        	\end{equation*}

        	Ahora sea (a,b,c) coordenadas de una esfera, entonces $a^2 + b^2 + c^2 = 1$.
        	Definimos $n = \left| \sqrt{\frac{a^2}{a^2 + b^2}}\right| $ y 
        	$b = \left|\sqrt{\frac{b^2 + a^2}{a^2 + b^2 + c^2}}\right| = \left|\sqrt{b^2 + a^2}\right|$
        	luego , $n\in \left[0,1\right]$ y $m\in \left[0,1\right]$, 
        	por tanto existe $\theta$ y $\phi$ tal que $n = \cos\theta$ y $m = \sin\phi$ luego:
        	\begin{equation*}
        	\begin{split}
        		(\sin\phi\cos\theta, \sin\phi\sin\theta, \cos\phi) &= (mn,\ m\sqrt{1-n^2},\ \sqrt{1-m^2}) \\
        														   &= (|\sqrt{a^2}|,\ |\sqrt{b^2}|, \ |\sqrt{c^2}|) \\
        														   &= (a,\ b,\ c)
        	\end{split}
        	\end{equation*} 	

        %Ejercicio 5%
        \item Determinar el determinante del Jacobiano para las coordenadas esféricas.\\\\
        Tenemos que:\\
        $x = \rho sin\phi cos\theta$\\ 
        $y = \rho sin\phi sin\theta$\\ 
        $z = \rho cos\phi$\\\\
        Ahora solo nos resta calcular el determinante y aplicar el valor absoluto:\\
        \[
        J =
        \begin{vmatrix}
        \frac{\partial}{\partial \rho} \rho sin\phi cos\theta & 
        \frac{\partial}{\partial \theta} \rho sin\phi cos\theta & 
        \frac{\partial}{\partial \phi} \rho sin\phi cos\theta \\ 
        \frac{\partial}{\partial \rho} \rho sin\phi sin\theta & 
        \frac{\partial}{\partial \theta} \rho sin\phi sin\theta & 
        \frac{\partial}{\partial \phi} \rho sin\phi sin\theta \\
        \frac{\partial}{\partial \rho} \rho cos\phi & 
        \frac{\partial}{\partial \theta} \rho cos\phi & 
        \frac{\partial}{\partial \phi} \rho cos\phi
        \end{vmatrix}
        \] 

		\[        
        =
        \begin{vmatrix}
        sin\phi cos\theta & 
        - sin\phi sin\theta & 
        \rho cos\phi cos\theta \\ 
        sin\phi sin\theta & 
        \rho sin\phi cos\theta & 
        \rho cos\phi sin\theta \\
        cos\phi & 
        0 & 
        - \rho sin\phi
        \end{vmatrix}
        \] 
        
		\[=  cos \phi 
        \begin{vmatrix}
        - \rho sin \phi sin \theta &
        \rho cos \phi cos \theta \\
        \rho sin \phi cos \theta &
        \rho cos \phi sin \theta
        \end{vmatrix} 
        - \rho sin \phi 
        \begin{vmatrix}
        sin \phi cos \theta &
        \rho sin \phi sin \theta \\
        sin \phi sin \theta &
        \rho sin \phi cos \theta
        \end{vmatrix}
        \]
      	$= - \rho^2 cos^2 \phi sin \phi sin^2 \theta - \rho^2 cos^2 \phi sin \phi cos^2 \theta - \rho^2 sin^3 \phi cos^2 \theta - \rho^2 sin^3 \phi sin^2 \theta$\\\\
      	$= - \rho^2 cos^2 \phi sin \phi - \rho^2 \sin^3 \phi$\\\\
      	$= - \rho^2 sin \phi (cos^2 \phi + sin^2 \phi)$\\\\
      	$= - \rho^2 sin \phi$\\\\
      	Al aplicar el valor absoluto, tenemos:\\\\
      	$\rho^2 sin \phi$\\\\
      	$\therefore J = \rho^2 sin \phi$  es el determinante del jacobiano.\\\\
        
        %Ejercicio 6%
        \item Usamos cambio a coordenadas polares para obtener 
        	  $x = r\cos\theta$, $y = \sin\theta$ y el determinante 
        	  Jacobiano que en este caso es $r$.

        	  Obtenemos $$\int\int rf(r\cos\theta, r\sin\theta)dr\ d\theta$$,
        	  a su vez $x^2 + y^2 = r^2\cos^2\theta + r^2\sin^2\theta = r^2$.

        	  Obtenemos los límites de intersección,
        	  Al ser D el disco unidad, tenemos: $0 \leq r\leq 1, 0 \leq \theta\leq 2\pi$.

        	  \begin{equation*}
        	  \begin{split}
        	  	\int_0^1\int_0^{2\pi} re^{r^2} d\theta\ dr 
        	  	&= \int_0^1\int_0^{2\pi} re^{r^2} d\theta\ dr \\
        	  	&= \int_0^1 re^{r^2} \int_0^{2\pi}  d\theta\ dr \\
        	  	&= \int_0^1 re^{r^2} 2\pi\ dr \\
        	  	&= 2\pi \int_0^1 re^{r^2}\ dr \\
        	  \end{split}
        	  \end{equation*}

        	  usando $t = r^3, du = 2r\ dr$
        	  obtenemos $2\pi/2(e-1) = \pi(e-1).$
        
        %Ejercicio 7%
        \item Sea $T(u,v) = (x(u,v),y(u,v))$ la aplicación definida por $T(u,v) = (4u,2u+3v)$. Sea $D^*$ el rectangulo $[0,1] \times [1,2]$. Hallar $D=T(D^*)$ \\ y calcular 

            \begin{enumerate}
                \item $\iint_D{xy}dxdy$
                \item $\iint_D{(x-y)}dxdy$
            \end{enumerate}

            \textbf{Solución:}

            Sean las rectas:

            \begin{tabular}{ll}
                $c_1 = (t,1)$ & $0 \leq t \leq 1$ \\
                $c_2 = (t,2)$ & $0 \leq t \leq 1$ \\
                $c_3 = (0,t)$ & $1 \leq t \leq 2$ \\
                $c_4 = (1,t)$ & $1 \leq t \leq 2$ \\
            \end{tabular}

            parametrizaciones de los lados del rectangulo $D^*$. \\
            Al componer las parametrizaciones de los lados del rectángulo con la transformacion tenemos las parametrizaciones de $D$:

            \begin{minipage}[b]{0.5\textwidth}
                \begin{tabular}{ll}
                    $T(c_1) = (4t,2t + 3)$ & $0 \leq t \leq 1$ \\
                    $T(c_2) = (4t,2t + 6)$ & $0 \leq t \leq 1$ \\
                    $T(c_3) = (0,3t)$ & $1 \leq t \leq 2$ \\
                    $T(c_4) = (4,2 + 3t)$ & $1 \leq t \leq 2$ \\
                \end{tabular}

                \vspace{0.3cm}
                Por lo tanto: $D$ es la región tal que: 

                \begin{tabular}{r@{ $\leq$ }c@{ $\leq$ }l}
                    $0$ & $u$ & $4$ \\ 
                    $\frac{u}{2} +3$ & $v$ & $\frac{u}{2}+6$.
                \end{tabular}

                \vspace{0.3cm}
                Y su jacbiano es: $\frac{\partial (x,y)}{\partial (u,v)}= 
                \begin{vmatrix}
                    4 & 0 \\
                    2 & 3
                \end{vmatrix}
                = 12$
                \vspace{0.5cm}

            \end{minipage}
            \begin{minipage}[b]{0.5\textwidth}
                \begin{tikzpicture}[scale=0.5, every node/.style={scale=0.7}]

                    \draw [<->] (0,4) -- (0,0) -- (2,0);
                    \node [above] at (0,4) {y};
                    \node [right] at (2,0) {x};

                    \draw [domain=0:1,variable=\t] plot ({\t},{1});
                    \draw [domain=0:1,variable=\t] plot ({\t},{2});
                    \draw [domain=1:2,variable=\t] plot ({1},{\t});

                    \shade [top color=white, bottom color=gray] (0,1) rectangle (1,2);
                    \node [above] at (0.5,1.2) {$D^*$};

                    \draw [->] (4,4) to [bend left] (8,4);
                    \node [above] at (6,4.7) {$T$};

                    \draw (1,0) -- (1,-0.2);
                    \node [below right] at (1,0) {$1$};
                    \draw [densely dotted] (1,0) -- (1,1);

                    \draw (0,1) -- (-0.2,1);
                    \node [above left] at (0,1) {$1$};

                    \draw (0,2) -- (-0.2,2);
                    \node [above left] at (0,2) {$2$};

                    \draw [<->] (10,9) -- (10,0) -- (15,0);
                    \node [above] at (10,9) {v};
                    \node [right] at (15,0) {u};

                    \draw [domain=0:1,variable=\t] plot ({4*\t +10},{2*\t + 3});
                    \draw [domain=0:1,variable=\t] plot ({4*\t +10},{2*\t + 6});
                    \draw [domain=1:2,variable=\t] plot ({4 +10},{2 + 3*\t});

                    \shade [top color=white, bottom color=gray] (10,3) -- (14,5) -- (14,8) -- (10,6);
                    \node [above] at (12,5.3) {$D$};

                    \draw (10,3) -- (9.8,3);
                    \node [above left] at (10,3) {3};

                    \draw (10,5) -- (9.8,5);
                    \node [above left] at (10,5) {5};
                    \draw [densely dotted] (14,5) -- (10,5);

                    \draw (10,6) -- (9.8,6);
                    \node [above left] at (10,6) {6};

                    \draw (10,8) -- (9.8,8);
                    \node [above left] at (10,8) {8};
                    \draw [densely dotted] (14,8) -- (10,8);

                    \draw (14,0) -- (14,-0.2);
                    \draw [densely dotted] (14,5) -- (14,0);
                    \node [below right] at (14,0) {4};


                    \node [below] at (12.5,4) {$T(c_1)$};
                    \node [above] at (11.5,7) {$T(c_2)$};
                    \node [right] at (10,4.5) {$T(c_3)$};
                    \node [right] at (14,6.5) {$T(c_4)$};

                \end{tikzpicture}
            \end{minipage}


            \begin{minipage}[t]{0.5\textwidth}
                \vspace{0.5cm}
                $\iint_D{xy}\,dxdy = \\
                =\int_0^4 \int_{\frac{u}{2}+3}^{\frac{u}{2}+6} (4u)(2u+3v)(12)\,dvdu = \\
                =12\int_0^4 \int_{\frac{u}{2}+3}^{\frac{u}{2}+6} (8u^2 +12uv)\,dvdu =  \\
                =12\int_0^4 (8u^2v + 6uv^2)\big|_{\frac{u}{2}+3}^{\frac{u}{2}+6}\,du = \\
                =12\int_0^4 (42u^2+162u)\,du = \\
                =12(\frac{42u^3}{3}+\frac{162u^2}{2})\big|_0^4 = \\ 
                =12(14u^3 + 81u^2)\big|_0^4 = \\
                =12(14*64 + 81*16) = \\
                =12(896+1296) = \\
                =12(2192) = \\
                =26304$ 

            \end{minipage}
            \begin{minipage}[t]{0.5\textwidth}
                \vspace{0.5cm}
                $\iint_D{(x-y)}\,dxdy = \\
                \int_0^4 \int_{\frac{u}{2}+3}^{\frac{u}{2}+6} (4u-2u-3v)(12)\,dvdu = \\
                12\int_0^4 \int_{\frac{u}{2}+3}^{\frac{u}{2}+6} (2u-3v)\,dvdu =  \\
                12\int_0^4 (2uv-\frac{3v^2}{2})\big|_{\frac{u}{2}+3}^{\frac{u}{2}+6}\,du =  \\
                12\int_0^4 \frac{1}{2}(39u+135)\,du \\
                6(\frac{39u^2}{2}+135u)\big|_0^4 = \\
                6(\frac{39*16}{2}+135*4) = \\
                6(\frac{624}{2}+540) = \\
                6(312+540) = \\
                6(852) = \\
                5112$

                \vspace{1cm}

            \end{minipage}
        
        %Ejercicio 8%
        \item \textbf{[Orlando]}
        
        %Ejercicio 9%
        \item \textbf{[Alex]}

        %Ejercicio 10%
        \item \textbf{[Alex]}                                      
    \end{enumerate}
        
\end{document}
