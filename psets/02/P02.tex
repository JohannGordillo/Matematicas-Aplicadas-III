\documentclass[10pt,letterpaper,fleqn]{article}

\usepackage[utf8]{inputenc}
\usepackage[spanish,es-nodecimaldot]{babel}
\usepackage{amsmath}
\usepackage{amssymb}
\usepackage{multicol}
\usepackage{graphicx}
\usepackage{amsthm}

\usepackage[dvipsnames]{xcolor}
\usepackage[most]{tcolorbox}

\usepackage{tabu}

\usepackage{pgfplots}
\pgfplotsset{width=10cm,compat=1.9}

\usepackage{mathtools}
\usepackage{tikz}
\usetikzlibrary{trees,positioning}

\usepackage[top=1in, bottom=1in, left=1in, right=1in]{geometry}


\begin{document}

% Portada
\begin{titlepage}
    \centering

    {\scshape\LARGE Universidad Nacional Autónoma de México \par}

    \vspace{1cm}
    
    {\scshape\Large Facultad de Ciencias\par}

    \vspace{1.5cm}

	% Logo de la UNAM.
    \begin{center}
        \includegraphics[scale=.17]{../../assets/img/logo.png}
    \end{center}

    \vspace{1 cm}

    {\LARGE Tarea 2 \par}
    
    \vspace{.3cm}
    
    {\huge\bfseries Matemáticas Aplicadas para las Ciencias III \par}

	\vspace{1.3cm}

    \large{\itshape{Johann Ramón Gordillo Guzmán}} \small{ - 418046090} \\
    
    \vspace{1cm}
	
	\large{\itshape{Luis Erick Montes Garcia}} \small{ - 419004547} \\    
    
	\vspace{1cm}    
    
    \large{\itshape{Ledesma Rincon Orlando}} \small{ - 419003234} \\

	\vspace{1cm}    
    
    \large{\itshape{Raymundo Méndez García}} \small{ - 113001958} \\

	\vspace{1cm}    
    
    \large{\itshape{Alex Gerardo Fernández Aguilar}} \small{ - 314338097} \\
    
    \vspace{0.3cm}

    \vfill

    Tarea presentada como parte del curso de
    \textbf{Matemáticas para las Ciencias Aplicadas III}
    impartido por el profesor \textbf{Zeús Alberto Valtierra Quintal}. \par
    \vspace{0.2cm}
    {\large 20 de Septiembre del 2019\par}
    \vspace{0.3cm}
    \footnotesize{\textbf{Link al código fuente:} https://github.com/JohannGordillo/Mates-lll-Tarea-2}
\end{titlepage}


    \begin{enumerate}

        %Ejercicio 1%
        \item \textbf{[Raymundo]}

        %Ejercicio 2%
        \item Sea $D^* = [0,1]$ x $[0, 1]$ y defínase $T$ en $D^*$ mediante $T(u, v) = (-u^2 - 4u, v).$\\
        Hallar la imagen $D$.\\
        ¿Es $T$ inyectiva?\\
        
        Tenemos que $D^* = [0,1]$ x $[0,1]$.\\
        Podemos analizar la frontera de la región y ver cómo se comporta la transformación al ser aplicada sobre éstas curvas:\\\\
        $c_1(t) = (t, 0)$\\
        $c_2(t) = (1, t)$\\
        $c_3(t) = (t, 1)$\\
        $c_4(t) = (0, t)$\\\\
        Al aplicar la transformación obténemos las rectas:\\
        $\gamma_1(t) = T(c_1(t)) = T(t, 0) = (-t^2 + 4t, 0)$\\
        $\gamma_2(t) = T(c_2(t)) = T(1, t) = (-1 + 4, t) = (3, t)$\\
		$\gamma_3(t) = T(c_3(t)) = T(t, 1) = (-t^2 + 4t, 1)$\\        
        $\gamma_4(t) = T(c_4(t)) = T(0, t) = (0, t)$\\
        con $t \in [0,1]$.\\\\
        $\Rightarrow \gamma_1$ es la recta del punto $(0,1)$ al punto $(3,1)$.\\
        $\Rightarrow \gamma_2$ es la recta del punto $(0,0)$ al punto $(0,1)$.\\
        $\Rightarrow \gamma_3$ es la recta del punto $(3,0)$ al punto $(3,1)$.\\
        $\Rightarrow \gamma_4$ es la recta del punto $(0,0)$ al punto $(3,0)$.\\
        Lo que obténemos son cuatro rectas que forman la región $D = [0, 3]$ x $[0, 1]$\\\\       
        Únicamente falta ver que la transformación sea una inyección. Analizando el comportamiento de la transformación y las regiones $D^*$ y $D$, podemos intuir que sí lo es. Pero hay que dar una demostración formal:
        \begin{proof}
        \vspace{0.2cm}
        Sean $(u,v), (u', v') \in D^*$.\\
        Supongamos que $T(u, v) = T(u', v')$.\\
        $\Rightarrow (-u^2 + 4u, v) = (-(u')^2 + 4u', v')$.\\
        $\therefore -u^2 + 4u = -(u')^2 + 4u'$ y adeáas $v = v'$.\\\\
        Como $v = v'$, falta probar que $u = u'$.\\
        Despejando: $-u^2 + 4u + (u')^2 - 4u' = 0$\\
        Aplicando la fórmula general para ecuaciones de segundo grado con variables\\
        $a = 1$\\
        $b = -4$\\
        $c = -u^2 + 4u$\\
        obténemos:\\
        $u' = \frac{4 \pm \sqrt{16 - 4(-u^2 + 4u)}}{2} = 2 \pm \frac{\sqrt{16 + 4 u^2 - 4u}}{2} = 2 \pm \frac{sqrt{4(u-2)^2}}{2}$\\\\
        $\therefore u' = u$ o $u' = 4 - u$.\\\\
        Pero sabemos que $u', u \in [0,1]$, por lo que no es posible que $u' = 4-u$\\
        $\therefore u = u'$\\
        $\therefore (u, v) = (u', v')$\\\\
        $\therefore T$ es una inyección.\\
		\end{proof}          
        
        %Ejercicio 3%
        \item Sea $D^* = [0,1]$ x $[0, 1]$ y defínase $T$ en $D^*$ mediante $T(x^*, y^*) = (x^*y^*, x^*).$\\
        Hallar la imagen $D$.\\
        ¿Es $T$ inyectiva?\\
        Si no lo es, ¿se puede quitar un subconjunto a $D^*$ para que $T$ sea inyectiva?\\\\
        Tenemos que $D^* = [0,1]$ x $[0, 1]$, que es la misma región que en el ejercicio anterior, por lo que obténemos las mismas rectas en la frontera:\\\\
        $c_1(t) = (t, 0)$\\
        $c_2(t) = (1, t)$\\
        $c_3(t) = (t, 1)$\\
        $c_4(t) = (0, t)$\\\\
        Al aplicar la transformación obténemos las rectas:\\
        $\gamma_1(t) = T(c_1(t)) = T(t, 0) = (t(0), t) = (0, t)$\\
        $\gamma_2(t) = T(c_2(t)) = T(1, t) = (1(t), t) = (t, 1)$\\
		$\gamma_3(t) = T(c_3(t)) = T(t, 1) = (t(1), t) = (t, t)$\\        
        $\gamma_4(t) = T(c_4(t)) = T(0, t) = (0(t), 0) = (0, 0)$\\
        con $t \in [0,1]$.\\\\
        $\Rightarrow \gamma_1$ es el $eje-y$.\\
        $\Rightarrow \gamma_2$ es la recta $y = 1$.\\
        $\Rightarrow \gamma_3$ es la recta $y = x$.\\
        $\Rightarrow \gamma_4$ es el origen $(0, 0)$.\\
        Lo que obténemos son tres rectas que forman la región triángular $D$ con vértices $(0,0)$, $(0,1)$ y $(1, 1)$.\\\\   
        Únicamente falta ver que la transformación sea una inyección. Analizando el comportamiento de la transformación y las regiones $D^*$ y $D$, podemos intuir que no lo es, ya que todos los puntos de la recta $c_4$ son mapeados al origen por la transformación $T$.\\\\
        Sin embargo, podemos hacer que la transformación sea inyectiva eliminando únicamente a la recta $c_4$ del dominio de la transformación.            

        %Ejercicio 4%
        \item \textbf{[Erick]}

        %Ejercicio 5%
        \item \textbf{[Raymundo]}
        
        %Ejercicio 6%
        \item \textbf{[Erick]}
        
        %Ejercicio 7%
        \item \textbf{[Orlando]}  
        
        %Ejercicio 8%
        \item \textbf{[Orlando]}
        
        %Ejercicio 9%
        \item \textbf{[Alex]}

        %Ejercicio 10%
        \item \textbf{[Alex]}                                      
    \end{enumerate}
        
\end{document}
